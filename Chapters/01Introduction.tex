\section{Prelim intro}
\label{sec:research_questions}
The project: ''GNSS-R: Formation flying of small satellites'' is a comparative study that investigates the accuracy of orbital propagation methods used within NTNU SmallSatLab's GNSS-R mission studies over longer time horizons. The analysis focuses on translational dynamics and compares two distinct simulation frameworks: the baseline SGP4 propagator implemented through the Skyfield library, and an extended high-fidelity propagator implemented in Basilisk. The latter enables the inclusion of additional perturbation models, such as Earth's gravity gradient, atmospheric drag, solar radiation pressure, and third-body pull, whose impact on long-term orbit prediction is assessed relative to the SGP4 baseline. To guide this investigation, the project addresses three key research questions:
\begin{itemize}
    \item What is the intrinsic numerical difference between the Skyfield-SGP4 and the Basilisk frameworks when configured identically?
    \item How does the introduction of extended perturbation models influence the divergence between the frameworks over time? 
    \item How do differences in simulated translational states affect formation-keeping delta-V analyses?
 \end{itemize}
Answering these questions will support the project's goal of guaranteeing the reliability and accuracy of orbital propagation tools used for GNSS-R small-satellite formation studies in future mission design.




\section{Background \& Motivation}

% gnss as a signal of oppertunity due to the global GNSS coverage
\gls{gnss} have become deeply embedded into our modern infrastructure with multiple constellations, such as GPS, GLONASS, Galileo and BeiDou, ensuring that four satellites are observable from the surface at any given time, thus providing worldwide coverage \cite{nasa_earthdata_gnss}. The sheer number of operational satellites that broadcast \gls{gnss} signals makes it one of the most powerful signals of opportunity for remote sensing, enabling monitoring without the need for onboard transmitters.
% Provide source for the final claim

\begin{figure}[H]
    \centering
    \includegraphics[width=0.5\textwidth]{Figures/01introduction/decreasing launch cost.png}
    \caption{Launch cost per kilogram to LEO versus first launch date, courtesy of \cite{jones2018launch_cost}}
    \label{fig:launch_cost} 
\end{figure}
 
% Launches has become cheaper, making it possible for smaller research gropus to implement space-based systems
At the same time, the space sector has recently undergone rapid changes. Launch costs have fallen significantly due to government outsourcing of launch providers to private companies, and the competitive environment it produced. This is clearly shown in figure \ref{fig:launch_cost}, where the per-kilogram launch cost has hit an all-time low with the deployment of SpaceX rockets \textit{Falcon9} and \textit{Falcon Heavy} \cite{jones2018launch_cost}. As a result, it has become increasingly feasible for universities, research groups and smaller nations to deploy space-based systems dedicated to leveraging \gls{gnss} signals for remote sensing applications. 

% Jamming and spoofing has increased recently, making it more valuable to locate the sources
In parallel with this growth, there has also been an increase in the frequency of reported intentional \gls{gnss} interference to disrupt aviation operations. The interference can be divided into separate categories characterized by their difference in method and effect: \textit{Jamming} refers to the relative high-power emission of \gls{rf} noise to block \gls{gnss} signals, while \textit{spoofing} is the transmission of signals that resemble those from real satellites, misleading the \gls{gnss} receiver to think it has a different position \cite{ffi_factsheet_2024}. The \gls{ffi} states that: "In Norway's northernmost county of Finnmark, close to the Russian border, loss of \gls{gnss} signals due to jamming occurs nearly daily. [\dots] The number of disruptions has increased significantly since Russia's invasion of Ukraine in 2022." \cite{ffi_jamming_2024}. Data from \textit{gpsjam.org} suggest that similar disruptions have been reported across Eastern Europe, the Middle East, and the Arctic \cite{gpsjam_2025}. These events highlight a broader geopolitical trend: \gls{gnss} interference is becoming more common. This places increased strategic value on systems that can detect and localize jamming and spoofing events.

% gnss-r as a promising technique for jamming/spoofing detection, and connection to SmallSat Lab
\gls{gnss-r} offers a promising technique for addressing these challenges. According to Jin and Komjathy \cite{jin2010gnss-R}, Hall and Cordey \cite{hall_cordey_1988} were the first to propose that GNSS L-band reflections could be exploited in a bistatic radar configuration to retrieve ocean surface properties, laying the foundation for modern \gls{gnss-r} techniques. Recent theory suggest that the same bistatic configuration can be used for surveillance of the \gls{gnss} spectrum, where multiple \gls{gnss-r} satellites in formation can detect and localize sources of jamming and spoofing. This emerging use-case is one of the central research goals of NTNU SmallSat Lab's \gls{gnss-r} project, which aim to develop methods for this process \cite{ntnu_smallsat_gnssr_2025}. The project currently investigates this concept through its \gls{gnss-r} satellite mission study, exploring formations geometries, orbit configurations and system designs that would enable robust detection of \gls{gnss} interference sources and ships. 
% Might move the mentioning of SmallSat Lab to after the motication

% why gnss-r is dependent on accurate orbital propagation
Accurate knowledge of spacecraft position and velocity is essential for the performance of any \gls{gnss-r} system. In such systems, regardless of application, the reflected \gls{gnss} signal is interpreted by comparing its time delay and Doppler shift relative to the direct signal. This process is highly sensitive to the specular reflection point's geolocation, which is given by the geometric relationship between the \gls{gnss} transmitter, the reflecting surface and the receiver. Even small errors in the satellite's propagated state shifts the estimated specular point by kilometers. Such errors can lead to misinterpretations of the delay-Doppler maps, for instance if the predicted specular point is over ocean when the true specular point occurs over land \cite{gleason2020cygnss}. 

In real missions, the satellite orbit and relative geometry within satellite formations will naturally drift over time due to orbital perturbations. During mission design, simulation tools are used to predict the satellites motion. If the simulation does not accurately model the relevant perturbations, its predicted state evolution will diverge from physically realistic behavior, eventually yielding incorrect specular point locations and misleading conclusions about the performance of the GNSS-R system. Furthermore, inaccuracies in position and velocity directly translates to uncertainty in later analyses, like coverage analysis, or formation-keeping delta-V estimation. For these reasons, a high-fidelity propagator is necessary to ensure realistic and reliable evaluations of the \gls{gnss-r} system performance.

\begin{comment}
    * The wider context of why GNSS-R has become an attractive technology
        - The increasing use of satellite constellations in earth observation and RF sensing
        - Trends in small satellites and CubeSats
        - (Consider expanding on why high-fidelity simulations are important for these missions)
        - Why GNSS-R is emerging as a technique with significant potential
    
    * Narrow down to NTNU SmallSatLab's GNSS-R project. The specialization project's relation to it

    * Why formation geometry is important for GNSS-R performance
        - Especially important for expected ISL bitrate and the GNSS-R measurement resolution 

    * An explanation of why simulation accuracy matters:
        - More accurate estimations of energy required to maintain formations
        - Higher accuracy GNSS-R coverage, pointing and constellation estimations

    * Describe the currently used GNSS-R orbital simulation framework, and state explicitly that its accuracy is unverified for satellite formation-drift studies
        - Specify propagation method, what it models, what it lacks and why that is problematic for GNSS-R mission planning

    * Tie to the need for a more in-depth comparison
        - Expand on the potential consequences of unverified accuracy. 

[Note to self]
- Make sure the formation geometry importance leads into the simulation accuracy importance. The flow should be:
    1. GNSS-R performance depends on specific formation geometry
    2. Formation geometry evolves due to orbital perturbations
    3. Therefore, high-fidelity simulation is required to estimate drift and formation control needs. 

- Don't slam the current implementation too hard. Use claims from Skyfield to back up its SGP4 accuracy
\end{comment}


\section{Problem Definition}


\begin{comment}
    * Define the core problem: How do different orbital propagation methods and perturbation effects influence satellite formations over time? 

    * Research question
        - Numerical precision of same-configuration simulations:   
            What is the numerical difference between the two simulation frameworks with the exact same simulation configuration? (same disturbances modelled, same integration time, same everything.)
        - Extended perturbation effects:   
            How does including additional orbital perturbations effect the numerical difference between the two simulation frameworks over time?
        - Computational performance:   
            How does the computational time differ between the simulation frameworks?

    * Sub-tasks (I don't feel like this section is really necessary)
        - Model orbital disturbance effects

        - Implement the two simulation frameworks 
\end{comment}


\section{Overall Method}
\begin{comment}
    * This needs to be filled out when I have more information, but from chat:

        * Introduce methodological approach: deterministic numerical simulation, model-by-model isolation, etc. Examples that COULD fit:
            - "comparative simulation study"
            - "Controlled scenario with fixed ICs"
            - "Pairwise comparison under identical perturbation models"

        * State clearly that both simulators will be initialized identically

        * Describe the experiment design

    * Describe the tools used for simulator design, and result verification
\end{comment}


\section{Project Scope \& Limitations}
\begin{comment}
    * Scope definition in terms of simulation parameters/configuration
        - Time horizon
        - Number of satellites
        - Formation type / orbital configuration
        - Orbital perturbations included
        - Propagation models chosen with corresponding python astrodynamics tool

    * Clarify limitations:
        - No attitude modelling
        - No GNSS-R modeling of any kind
        - No formation control 
        - Orbital perturbations excluded 
        - Assumptions and perturbation model simplifications
        - Conclusions are limited to the tested scenarios and may not generalize into all LEO missions

    * Connect limitations to the Master's thesis

[Notes for myself in this section:]
- Be specific in listing all above parameters. Avoid general terms, and use their concrete names
- Specify what the project does not attempt to solve 

\end{comment}


\section{Report Structure}
\begin{comment}
    * Brief overview of what each chapter contains

    * This will be populated thoroughly when the sections are written...
\end{comment}


