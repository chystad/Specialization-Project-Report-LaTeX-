% Some intoduction that ties the literature review to the theory chapter

% \section{Classical Orbital Elements}
% Only really relevent for understanding TLEs for SGP4 propagation. Prioritize last Check if needed.

\section{Reference Frames}
% Tommy mentioned that it is extremely important to be precise in the reference frame definitions and usage later on

A reference frame is uniquely described by $\mathcal{F}^{\,\textit{rf}}:\{\mathcal{O}_{\textit{rf}}\,; \hat{x}_{\textit{rf}}\,, \hat{y}_{\textit{rf}}\,, \hat{z}_{\textit{rf}}\}$ where $\mathcal{O}_{\textit{rf}}$ is the origin, and $\{\hat{x}_{\textit{rf}}\,, \hat{y}_{\textit{rf}}\,, \hat{z}_{\textit{rf}}\}$ denote the dextral orthogonal unit vectors \cite{Grotte2020SpacecraftAttitude}. In this work, mainly two reference frames are used: the \gls{eci} frame and the \gls{rtn} frame, which is illustrated by figure \ref{fig:frame_def}. The \gls{eci} frame is the inertial frame used to express all absolute satellite motion about the Earth, while the \gls{rtn} frame provides a satellite-centered frame of reference suited for expressing relative motion within a formation. Lastly, because gravitational and geophysical models are naturally defined in an Earth-fixed coordinate system, a third frame, the \gls{ecef} frame, is also introduced.

\begin{figure}[H]
    \centering
    \includegraphics[width=0.6\textwidth]{Figures/03Theory/Frames.png}
    \caption{Illustration of the Earth-Centered Inertial ($\mathcal{F}^{\,i}$) and Radial-Transverse-Normal ($\mathcal{F}^{\,r}$) frames used in this project. The unit vectors have different length for illustrative purposes. Adapted from \cite{eci2014, Metz2020CollisionAvoidance}} 
    \label{fig:frame_def} 
\end{figure}


\subsection{Earth-Centered Inertial (ECI) Frame}
The \gls{eci} frame, $\mathcal{F}^{\,i}:\{\mathcal{O}_{i}\,; \hat{x}_{i}\,, \hat{y}_{i}\,, \hat{z}_{i}\}$, has its origin at the Earth's \gls{cm} and fixed principal axes relative to the celestial sphere. At any given time, $\hat{x}_{i}$ points towards the Vernal Equinox, per definition striking through the equator. $\hat{z}_{i}$ points along the Earth's axis of rotation through its geographical North-Pole, and $\hat{y}_{i}$ is defined using the right-hand rule. An illustration showing this principal axis configuration is shown in figure \ref{fig:frame_def}. The position of an arbitrary satellite, $s$, relative to the Earth's \gls{cm} can then be expressed by its principal components in the \gls{eci} frame with the following notation:
\begin{equation}
    r_{s/i}^i =  a\hat{x}_{i}^i + b\hat{y}_{i}^i + c\hat{z}_{i}^i = \begin{bmatrix}
        a & b & c
    \end{bmatrix}^T
\end{equation}
% TODO: define J2000 ECI


\subsection{Radial-Transverse-Normal (RTN) Frame}
The \gls{rtn} frame, $\mathcal{F}^{\,r}:\{\mathcal{O}_{r}\,; \hat{x}_{r}\,, \hat{y}_{r}\,, \hat{z}_{r}\}$, is defined with its origin at the  leader's \gls{cm} in a leader-follower satellite formation. Using $l$ to denote the leader satellite, the frame's orthogonal dextral unit vectors can be defined in the \gls{eci} frame as: 
\begin{equation}
    \hat{R}^i = \hat{x}_r^i = \frac{r^i_{l/i}}{||r^i_{l/i}||}, \qquad \hat{N}^i = \hat{z}_r^i = \frac{r^i_{l/i} \times v^i_{l/i}}{||r^i_{l/i} \times v^i_{l/i}||}, \qquad \hat{T}^i = \hat{y}_r^i = \hat{z}^i_r \times \hat{x}_r^i
\end{equation}
Here, $\hat{x}_r^i$ is the radial unit vector, pointing outwards along the leader's position vector. $\hat{z}_r^i$ points in the direction orthogonal to $\{ r^i_{l/i}, v^i_{l/i} \}$ and corresponds to the cross-track direction. Finally, $\hat{y}_r^i$ is defined through the right-hand rule, giving the along-track direction. An illustration of these definitions is presented in figure \ref{fig:frame_def}. Using this definition, the transformations to \gls{eci} from \gls{rtn} and back is given by:
\begin{equation}
    R^i_r = \begin{bmatrix}
        \hat{x}^i_r & \hat{y}^i_r & \hat{z}^i_r
    \end{bmatrix}, \qquad R^r_i = (R^i_r)^T
\end{equation}  

% \begin{equation}
%     R^i_r = \begin{bmatrix}
%         \hat{x}^i_r & \hat{y}^i_r & \hat{z}^i_r
%     \end{bmatrix}, \qquad R^r_i = (R^i_r)^T = \begin{bmatrix}
%         (\hat{x}^i_r)^T \\
%         (\hat{y}^i_r)^T \\
%         (\hat{z}^i_r)^T
%     \end{bmatrix}
% \end{equation}  


\subsection{Earth-Fixed Earth-Centered (ECEF) Frame}
The \gls{ecef} frame: $\mathcal{F}^{\,f}:\{\mathcal{O}_{f}\,; \hat{x}_{f}\,, \hat{y}_{f}\,, \hat{z}_{f}\}$, is a rotating terrestrial frame whose axes are fixed relative to the Earth's surface. Its origin $\mathcal{O}_{f}=\mathcal{O}_{i}$ is located at the Earth's \gls{cm}, while $\hat{z}_f = \hat{z}_i$ aligns with the planet's mean rotation axis pointing towards the geographical North-Pole. $\hat{x}_f$ intersects the equator and the prime meridian, and $\hat{y}_f$ fulfills the right-hand rule by pointing $90^\circ$ East longitude. Because the Earth is spinning, the transformation between the \gls{eci} and \gls{ecef} frames is inherently time-dependent. In this project, the rotation between the two frames is described using the formulation provided by Dr. Stryjewski \cite{Stryjewski2020coord_trans}, who expresses the Earth's rotation about the inertial $\hat{z}_i$-axis through a time-varying angle $\theta(t)$. The epoch at which the \gls{eci} and \gls{ecef} frames coincide is J2000, meaning that $\theta(t) = 0$ at 12:00 TT on 1 January 2000. Relative to this epoch, the rotation into \gls{ecef} at any later time $t$ is given by:
\begin{equation}
    R^f_i = \begin{bmatrix}
        \cos \theta(t) & -\sin \theta(t) &  0 \\
        \sin \theta(t) & \cos \theta(t) & 0 \\
        0 & 0 & 1
    \end{bmatrix}, \qquad R^i_f = (R^f_i)^T
\label{eq:eci2ecef}
\end{equation}
where the angle is computed as:
\begin{equation}
    \theta(t) = \omega (GMST(t) - (UT1(t)-UTC(t))), \qquad \omega = 0.261799387799149rad/h
\end{equation}

\section{Orbital Perturbations}

% \begin{figure}[H]
%     \centering
%     \includegraphics[width=0.5\textwidth]{Figures/03Theory/perturbation magnitudes.png}
%     \caption{Order of magnitude of various perturbations of a satellite orbit. Courtesy of \cite{Montenbruck2000}}
%     \label{fig:perturb_mag} 
% \end{figure}



\subsection{Spherical Harmonics} \label{sec:J2}

Assuming the Earth's mass is concentrated in the \gls{eci} frame's origin, the acceleration experienced by a spacecraft, $s$, can be modelled using Newton's gravitational law:
\begin{equation}
    \ddot{r}_{s/i} = - \frac{\mu}{||r_{s/i}||_2^3} r_{s/i} \iff \ddot{r}_{s/i} = \nabla U, \qquad \text{where } U = \mu\frac{1}{||r_{s/i}||_2}
\label{eq:newton_gravity}
\end{equation}
Here $\mu = GM$ is the Earth's gravitational parameter, and $\nabla U$ is an equivalent representation given by the gradient of the gravity potential. Other literature define $U$ with the opposite sign, but here $U>0$ is chosen to stay consistent with the convention used by Montenbruck, et al. Their literature explains that this model is inaccurate due to the Earth's mass being unhomogeneously distributed through its volume, and because its rotational velocity causes its shape to bulge around the equator \cite{Montenbruck2000}. They go on to present an extension of equation \ref{eq:newton_gravity} to account for these irregularities in the Earth's gravitational field, starting with a generalization of the gravity potential by summing up the contributions from all internal mass elements. With $\rho(r^f_{p/i})$ representing the density at some point $p$, and $\{r_{s/i}^f, r_{p/i}^f\}$ denoting the positions of the satellite and the point in \gls{ecef} respectively, the general gravity potential can be expressed as:
\begin{equation}
    U = G \int \frac{1}{||r_{s/i}^f - r_{p/i}^f||_2} dm= G \int \frac{1}{||r_{s/i}^f - r_{p/i}^f||_2} \rho(r_{p/i}^f) d^3 r_{p/i}^f
\label{eq:grav_acc}
\end{equation}
However, the density distribution is not accurately known, which makes the expression impractical. To evaluate the integral further, the fraction can be expanded using a series of Legendre polynomials. The general Legendre polynomial $P_{nm}$ of degree $n$ and order $m$ is defined as:
\begin{equation}
    P_{nm}(u) = (1-u^2)^{m/2} \frac{d^m}{du^m}P_n(u)
\end{equation}
Introducing the spacecraft's geocentric distance $r = ||r_{s/i}||_2$, the latitude $\phi^f$ and the longitude $\lambda^f$ allows the satellite's position in the \gls{ecef} frame to be expressed in spherical coordinates $\{ r, \phi^f, \lambda^f \}$. The superscript $^f$ is in this case used to indicate that the angles are Earth-fixed. Using these coordinates together with the associated Legendre polynomials, the gravity potential can be written in its general spherical harmonic form:
\begin{equation}
    U = \frac{\mu}{R} \sum_{n=0}^{\infty} \sum_{m=0}^{n} (C_{nm}V_{nm} + S_{nm}W_{nm}), \; \begin{cases}
        V_{nm} &= \left( \frac{R}{r} \right)^{n+1} P_{nm}(\sin \phi^f) \cdot \cos m \lambda^f \\
        W_{nm} &= \left( \frac{R}{r} \right)^{n+1} P_{nm}(\sin \phi*f) \cdot \sin m \lambda^f 
    \end{cases}
\label{eq:grav_pot_full}
\end{equation}
Here, $R$ denote the Earth's reference radius, and the geopotential coefficients $C_{nm}$ and $S_{nm}$ encode the internal mass distribution's effect on the satellite. Because the longitudinal mass variations are comparatively small, a common simplification is to set $m=0$, giving the so-called zonal coefficient. Under this assumption, $J_n = -C_n0$ and $W_{n0}=0$, which significantly simplifies the equation \ref{eq:grav_pot_full}. Using the zonal terms up to forth degree, the truncated gravitational potential can be computed from the Legendre polynomials and $J$-terms in table \ref{tab:geopotential_legendre}:


\begin{equation}
\begin{split}
    U (r, \phi^f, \lambda^f)&= \frac{\mu}{R} \sum_{n=0}^{4}(C_{n0} V_{n0} + S_{n0}W_{n0}) \\
    U (r, \phi^f)&= \frac{\mu}{R} \sum_{n=0}^{4}(C_{n0} V_{n0}) \\
     &= \frac{\mu}{R} \left[C_{00}V_{00} + C_{10}V_{10} + C_{20}V_{20} + C_{30}V_{30} + C_{40}V_{40} \right] \\
     &= \frac{\mu}{R} \left[ \left( \frac{R}{r} \right) -J_2\left( \frac{R}{r} \right)^3 P_2(\sin \phi^f) \right. \\
    & \qquad \quad \left.-J_3\left( \frac{R}{r} \right)^4 P_3(\sin \phi^f) -J_4\left( \frac{R}{r} \right)^5 P_4(\sin \phi^f) \right] \\
    U (r, \phi^f) &= \frac{\mu}{r} \left[ 1 -J_2\left( \frac{R}{r} \right)^2 P_2(\sin \phi^f) \right. \\
    & \qquad \quad \left.-J_3\left( \frac{R}{r} \right)^3 P_3(\sin \phi^f) -J_4\left( \frac{R}{r} \right)^4 P_4(\sin \phi^f) \right] \\
    U(r, \phi^f) &= \frac{\mu}{r} \left[ 1 - \sum_{n=2}^{4} J_n \left( \frac{R}{r} \right)^n P_n(\sin \phi^f) \right]
\end{split}
\end{equation}
Among the zonal coefficients, $J_2$ is by far the most significant, and dominates the non-spherical effects. It captures the Earth's oblateness and produces the well known secular perturbations in the orbital elements, which drifts the right ascension of the ascending node and rotation of the argument of perigee. The higher-degree $J_3$ and $J_4$ terms are several orders of magnitude less significant, but still produce a notable perturbation. $J_3$ induces long-period variations in eccentricity due to asymmetry between the northern and southern hemispheres, while $J_4$ refines the flatness modelling, and contributes to small corrections to the secular rates caused by $J_2$ \cite{vallado2001, Montenbruck2000}.
\begin{table}[h!]
\centering
\begin{tabular}{c c c}
\hline
$n$ & $J_n$ & $P_n(u)$ \\
\hline
0 & $-1$ & $1$ \\
1 & $0$ & $u$ \\
2 & $1.08263 \times 10^{-3}$ & $\tfrac{1}{2}(3u^2 - 1)$ \\
3 & $-2.53266 \times 10^{-6}$ & $\tfrac{1}{2}(5u^3 - 3u)$ \\
4 & $-1.61962 \times 10^{-6}$ & $\tfrac{1}{8}(35u^4 - 30u^2 + 3)$ \\
\hline
\end{tabular}
\caption{Zonal coefficients and associated Legendre polynomials up to fourth degree. Values courtesy of \cite{Zhong2013}.}
\label{tab:geopotential_legendre}
\end{table}
Inserting this truncated spherical harmonic expression into equation \ref{eq:grav_acc} yields the acceleration experienced by an arbitrary spacecraft, expressed in the \gls{ecef} frame. By computing the gradient, then applying the time-varying transformation in equation \ref{eq:eci2ecef}, the acceleration can then be transformed into the \gls{eci} frame:
\begin{equation}
    \ddot{r}^f_{s/i} = \nabla U(r, \phi^f) \;\;\Longrightarrow\;\; \ddot{r}^i_{s/i} = R^i_f \nabla U(r, \phi^f)
\end{equation}

\subsection{Atmospheric Drag} \label{sec:Atmospheric_Drag}

Atmospheric drag exponential density model from \cite{vallado2001, Montenbruck2000, Roscoe2014}
\[
\mathbf{a}_{\text{drag}} = -\tfrac{1}{2} C_D \, \frac{A}{m} \, \rho \, v_{\text{rel}}^{2} \, \hat{\mathbf{v}}_{\text{rel}},
\]

\begin{itemize}
    \item $C_D$ is the drag coefficient,
    \item $A$ is the effective cross-sectional area,
    \item $m$ is the spacecraft mass,
    \item $\rho$ is the atmospheric density at the spacecraft location,
    \item $\mathbf{v}_{\text{rel}}$ is the velocity of the spacecraft \textit{relative to the atmosphere},
    \item $v_{\text{rel}} = \lVert \mathbf{v}_{\text{rel}} \rVert$,
    \item $\hat{\mathbf{v}}_{\text{rel}} = \mathbf{v}_{\text{rel}} / v_{\text{rel}}$.
\end{itemize}

% Exponential density model
\[
\rho(h) = \rho_0 \exp\!\left( -\frac{h - h_0}{H} \right),
\]

where

\begin{itemize}
    \item $h$ is the altitude, $h = r - R_E$,
    \item $\rho_0$ is the reference density at altitude $h_0$,
    \item $H$ is the scale height.
\end{itemize}

% Full drag acceleration with exponential atmosphere
\[
\mathbf{a}_{\text{drag}}
= -\tfrac{1}{2} C_D \, \frac{A}{m} \, \rho_0
\exp\!\left( -\frac{h - h_0}{H} \right)
v_{\text{rel}}^{2} \, \hat{\mathbf{v}}_{\text{rel}}.
\]




\subsection{Solar Radiation Pressure} \label{sec:Solar_Pressure}

Radiation is modeled by using solar flux at one astronomical unit and scaling by distance from the sun relative to 1 AU. The solar flux at one AU is taken as:

\begin{equation}
    SF_{AU} = 1372.5398 \; \left[ \frac{W}{m^2} \right]
\end{equation}

The cannonball model assumes the spacecraft is a simple sphere. The radiation pressure at 1AU, pSR, can be taken as the solar flux divided by the speed of light.

\begin{equation}
    p_{SR} = \frac{SF_{AU}}{c} \; \left[ \frac{N}{m^2} \right]
\end{equation}

Then, a “scaling factor” can be determined. This “scaling factor” is equivalent to the magnitude of the solar radiation force divided by the distance between the spacecraft and the sun:

\begin{equation}
    \frac{|\mathbf{F}_{\text{radiation}}|}{|\mathbf{r}_{\text{sun}}|} = 
    - c_R \, p_{SR} \, A_d \, \frac{AU^2}{|\mathbf{r}_{\text{sun}}|^3}
    \; \left[ \frac{N}{m} \right]
\end{equation}

rsun is the vector from the spacecraft to the sun in the spacecraft body frame and cR is the reflectivity. This factor is then multiplied by the position vector from the spacecraft to the sun to get the force on the spacecraft due to solar radiation pressure.

\begin{equation}
    \mathbf{F}_{\text{radiation}} =
    \frac{|\mathbf{F}_{\text{radiation}}|}{|\mathbf{r}_{\text{sun}}|}
    \, \mathbf{r}_{\text{sun}} \; [N]
\end{equation}


\subsection{Third-body Perturbations}

Third-body gravitational pull from the sun and the moon. Formulations gathered from \cite{vallado2001, Montenbruck2000, Wertz1978, Roscoe2014}

Let $\mathbf{r}$ be the position of the spacecraft with respect to the Earth, and 
$\mathbf{r}_{3B}$ the position of the third body (Sun or Moon) with respect to the Earth, 
both expressed in the same inertial frame.

The standard third-body acceleration is

\[
\mathbf{a}_{3B}
    = \mu_{3B}
    \left(
        \frac{\mathbf{r}_{3B} - \mathbf{r}}{\lVert \mathbf{r}_{3B} - \mathbf{r} \rVert^{3}}
        \;-\;
        \frac{\mathbf{r}_{3B}}{\lVert \mathbf{r}_{3B} \rVert^{3}}
    \right),
\]

where

\begin{itemize}
    \item $\mu_{3B}$ is the gravitational parameter of the third body (Sun or Moon),
    \item the first term is the gravitational acceleration of the third body on the spacecraft,
    \item the second term subtracts the acceleration of the Earth due to the third body 
          (so the result is in the Earth-centred frame).
\end{itemize}

If you model both Sun and Moon, the total third-body acceleration is

\[
\mathbf{a}_{3B,\text{total}} = \mathbf{a}_{\text{Sun}} + \mathbf{a}_{\text{Moon}},
\]

each term computed using the same formula above with the appropriate 
$\mu_{\odot}$, $\mu_{\text{Moon}}$ and ephemerides.



\section{Numerical Orbit Propagation}
% How basilisk propagates the orbit


\section{Analytical Orbit Propagation - SGP4}
% How 