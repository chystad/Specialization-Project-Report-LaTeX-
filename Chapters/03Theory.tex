% Some intoduction that ties the literature review to the theory chapter

% \section{Classical Orbital Elements}
% Only really relevent for understanding TLEs for SGP4 propagation. Prioritize last Check if needed.

\section{Reference Frames}
% Tommy mentioned that it is extremely important to be precise in the reference frame definitions and usage later on

A reference frame is uniquely described by $\mathcal{F}^{\,\textit{rf}}:\{\mathcal{O}_{\textit{rf}}\,; \hat{x}_{\textit{rf}}\,, \hat{y}_{\textit{rf}}\,, \hat{z}_{\textit{rf}}\}$ where $\mathcal{O}_{\textit{rf}}$ is the origin, and $\{\hat{x}_{\textit{rf}}\,, \hat{y}_{\textit{rf}}\,, \hat{z}_{\textit{rf}}\}$ denote the dextral orthogonal unit vectors \cite{Grotte2020SpacecraftAttitude}. In this work, two reference frames are used: the \gls{eci} frame and the \gls{rtn} frame, which is illustrated by figure \ref{fig:frame_def}. The \gls{eci} frame is the inertial frame used to express all absolute satellite motion about the Earth, while the \gls{rtn} frame provides a satellite-centered frame of reference useful for expressing relative motion within a formation. 

\begin{figure}[H]
    \centering
    \includegraphics[width=0.6\textwidth]{Figures/03Theory/Frames.png}
    \caption{Definition of the Earth-Centered Inertial ($\mathcal{F}^{\,i}$) and Radial-Transverse-Normal ($\mathcal{F}^{\,r}$) frames used in this project. The RTN origin is centered in the leader's center of mass in a leader-follower satellite formation. The unit vectors have different length for illustrative purposes. Adapted from \cite{eci2014, Metz2020CollisionAvoidance}}
    \label{fig:frame_def} 
\end{figure}




\subsection{Earth-Centered Inertial (ECI) Frame}
The \gls{eci} frame, $\mathcal{F}^{\,i}:\{\mathcal{O}_{i}\,; \hat{x}_{i}\,, \hat{y}_{i}\,, \hat{z}_{i}\}$, has its origin at the Earth's \gls{cm} and fixed principal axes relative to the celestial sphere. At any given time, $\hat{x}_{i}$ points towards the Vernal Equinox, per definition striking through the equator. $\hat{z}_{i}$ points along the Earth's axis of rotation through its geographical North-Pole, and $\hat{x}_{i}$ is defined using the right-hand rule. An illustration showing this principal axis configuration is shown in figure \ref{fig:frame_def}. The position of an arbitrary satellite, $s$, relative to the Earth's \gls{cm} can then be expressed by its principal components in the \gls{eci} with the following notation:

\begin{equation}
    r_{s/i}^i =  a\hat{x}_{i}^i + b\hat{y}_{i}^i + c\hat{z}_{i}^i = \begin{bmatrix}
        a & b & c
    \end{bmatrix}^T
\end{equation}



\subsection{Radial-Transverse-Normal (RTN) Frame}
The \gls{rtn} frame, $\mathcal{F}^{\,r}:\{\mathcal{O}_{r}\,; \hat{x}_{r}\,, \hat{y}_{r}\,, \hat{z}_{r}\}$, is defined with its origin at the  leader's \gls{cm} in a leader-follower satellite formation. Using $l$ to denote the leader satellite, the frame's orthogonal dextral unit vectors can be defined in the \gls{eci} frame as: 
\begin{equation}
    \hat{R}^i = \hat{x}_r^i = \frac{r^i_{l/i}}{||r^i_{l/i}||}, \qquad \hat{N}^i = \hat{z}_r^i = \frac{r^i_{l/i} \times v^i_{l/i}}{||r^i_{l/i} \times v^i_{l/i}||}, \qquad \hat{T}^i = \hat{y}_r^i = \hat{z}^i_r \times \hat{x}_r^i
\end{equation}
Here, $\hat{x}_r^i$ is the radial unit vector, pointing outwards along the leader's position vector. $\hat{z}_r^i$ points in the direction orthogonal to $\{ r^i_{l/i}, v^i_{l/i} \}$ and corresponds to the cross-track direction. Finally, $\hat{y}_r^i$ is defined through the right-hand rule, giving the along-track direction. An illustration of these definitions is presented in figure \ref{fig:frame_def}. Using this definition, the transformations to \gls{eci} from \gls{rtn} and back is given by:
\begin{equation}
    R^i_r = \begin{bmatrix}
        \hat{x}^i_r & \hat{y}^i_r & \hat{z}^i_r
    \end{bmatrix}, \qquad R^r_i = (R^i_r)^T = \begin{bmatrix}
        (\hat{x}^i_r)^T \\
        (\hat{y}^i_r)^T \\
        (\hat{z}^i_r)^T
    \end{bmatrix}
\end{equation}  


\section{Orbital Perturbations}

\subsection{Spherical Harmonics} \label{sec:J2}

Gravitational potential with J2 - J4 from \cite{vallado2001, Montenbruck2000}

\begin{equation}
    U(r, \phi) = \frac{\mu}{r} \left[ 1 - \sum_{n=2}^{4} J_n \left( \frac{R_E}{r} \right)^n P_n(\sin \phi) \right]
\end{equation}

where:
\begin{itemize}
    \item $\mu$ is Earth's gravitational parameter
    \item $R_E$ is the Earth's equatorial radius
    \item $r = ||r||$ is the geocentric distance
    \item $\phi$ is the geocentric latitude
    \item $J_n$ are the zonal harmonic coefficients
    \item $P_n(\cdot)$ are the Legendre polynomials
\end{itemize}

\begin{equation}
    a_{grav} = -\nabla U(r)
\end{equation}

TODO: Explain the decreasing significance of J2, J3 and J4 terms, and how they affect the mean orbital elements
TODO: Add a table giving the numerical values for the J2-J4 terms

\subsection{Atmospheric Drag} \label{sec:Atmospheric_Drag}

Atmospheric drag exponential density model from \cite{vallado2001, Montenbruck2000, Roscoe2014}
\[
\mathbf{a}_{\text{drag}} = -\tfrac{1}{2} C_D \, \frac{A}{m} \, \rho \, v_{\text{rel}}^{2} \, \hat{\mathbf{v}}_{\text{rel}},
\]

\begin{itemize}
    \item $C_D$ is the drag coefficient,
    \item $A$ is the effective cross-sectional area,
    \item $m$ is the spacecraft mass,
    \item $\rho$ is the atmospheric density at the spacecraft location,
    \item $\mathbf{v}_{\text{rel}}$ is the velocity of the spacecraft \textit{relative to the atmosphere},
    \item $v_{\text{rel}} = \lVert \mathbf{v}_{\text{rel}} \rVert$,
    \item $\hat{\mathbf{v}}_{\text{rel}} = \mathbf{v}_{\text{rel}} / v_{\text{rel}}$.
\end{itemize}

% Exponential density model
\[
\rho(h) = \rho_0 \exp\!\left( -\frac{h - h_0}{H} \right),
\]

where

\begin{itemize}
    \item $h$ is the altitude, $h = r - R_E$,
    \item $\rho_0$ is the reference density at altitude $h_0$,
    \item $H$ is the scale height.
\end{itemize}

% Full drag acceleration with exponential atmosphere
\[
\mathbf{a}_{\text{drag}}
= -\tfrac{1}{2} C_D \, \frac{A}{m} \, \rho_0
\exp\!\left( -\frac{h - h_0}{H} \right)
v_{\text{rel}}^{2} \, \hat{\mathbf{v}}_{\text{rel}}.
\]




\subsection{Solar Radiation Pressure} \label{sec:Solar_Pressure}

Radiation is modeled by using solar flux at one astronomical unit and scaling by distance from the sun relative to 1 AU. The solar flux at one AU is taken as:

\begin{equation}
    SF_{AU} = 1372.5398 \; \left[ \frac{W}{m^2} \right]
\end{equation}

The cannonball model assumes the spacecraft is a simple sphere. The radiation pressure at 1AU, pSR, can be taken as the solar flux divided by the speed of light.

\begin{equation}
    p_{SR} = \frac{SF_{AU}}{c} \; \left[ \frac{N}{m^2} \right]
\end{equation}

Then, a “scaling factor” can be determined. This “scaling factor” is equivalent to the magnitude of the solar radiation force divided by the distance between the spacecraft and the sun:

\begin{equation}
    \frac{|\mathbf{F}_{\text{radiation}}|}{|\mathbf{r}_{\text{sun}}|} = 
    - c_R \, p_{SR} \, A_d \, \frac{AU^2}{|\mathbf{r}_{\text{sun}}|^3}
    \; \left[ \frac{N}{m} \right]
\end{equation}

rsun is the vector from the spacecraft to the sun in the spacecraft body frame and cR is the reflectivity. This factor is then multiplied by the position vector from the spacecraft to the sun to get the force on the spacecraft due to solar radiation pressure.

\begin{equation}
    \mathbf{F}_{\text{radiation}} =
    \frac{|\mathbf{F}_{\text{radiation}}|}{|\mathbf{r}_{\text{sun}}|}
    \, \mathbf{r}_{\text{sun}} \; [N]
\end{equation}


\subsection{Third-body Perturbations}

Third-body gravitational pull from the sun and the moon. Formulations gathered from \cite{vallado2001, Montenbruck2000, Wertz1978, Roscoe2014}

Let $\mathbf{r}$ be the position of the spacecraft with respect to the Earth, and 
$\mathbf{r}_{3B}$ the position of the third body (Sun or Moon) with respect to the Earth, 
both expressed in the same inertial frame.

The standard third-body acceleration is

\[
\mathbf{a}_{3B}
    = \mu_{3B}
    \left(
        \frac{\mathbf{r}_{3B} - \mathbf{r}}{\lVert \mathbf{r}_{3B} - \mathbf{r} \rVert^{3}}
        \;-\;
        \frac{\mathbf{r}_{3B}}{\lVert \mathbf{r}_{3B} \rVert^{3}}
    \right),
\]

where

\begin{itemize}
    \item $\mu_{3B}$ is the gravitational parameter of the third body (Sun or Moon),
    \item the first term is the gravitational acceleration of the third body on the spacecraft,
    \item the second term subtracts the acceleration of the Earth due to the third body 
          (so the result is in the Earth-centred frame).
\end{itemize}

If you model both Sun and Moon, the total third-body acceleration is

\[
\mathbf{a}_{3B,\text{total}} = \mathbf{a}_{\text{Sun}} + \mathbf{a}_{\text{Moon}},
\]

each term computed using the same formula above with the appropriate 
$\mu_{\odot}$, $\mu_{\text{Moon}}$ and ephemerides.



\section{Numerical Orbit Propagation}
% How basilisk propagates the orbit


\section{Analytical Orbit Propagation - SGP4}
% How 