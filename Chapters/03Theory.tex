% Some intoduction that ties the literature review to the theory chapter

\section{Classical Orbital Elements}
% Only really relevent for understanding TLEs for SGP4 propagation. Prioritize last

\section{Reference Frames}
Some general notation:
\begin{displaymath}
    R^{to}_{from}, \qquad r^{frame}_{to/from}
\end{displaymath}

\subsection{Inertial ECI frame}
$\mathcal{F}^i$. Origin in the Earth's center and frame orientation fixed relative to the stars. Differs from ECEF in that respect. x-axis permanently fixed in a direction relative to the celestial sphere z-axis 90 degrees angle to the equatorial plane and extends up through the geographical North-Pole. y-axis defined with right-hand-rule

\begin{figure}[H]
    \centering
    \includegraphics[width=0.5\textwidth]{Figures/03Theory/ECI.png}
    \caption{Earth Centered Inertial (ECI) coordinate system, courtesy of \cite{eci2014}}
    \label{fig:eci_def} 
\end{figure}


\subsection{RTN frame}
x-axis (R) outwards from the position vector (radial direction), z-axis (N) pointing towards the cross product between the position and velocity vectors (along-track direction), y-axis (T) completing the right-handed system (cross-track direction). Denoted by $\mathcal{F}^r$
\begin{equation}
    \hat{R}^i_r = \frac{r^i_{c/e}}{||r^i_{c/e}||}, \qquad \hat{N}^i_r = \frac{r^i_{c/e} \times v^i_{c/e}}{||r^i_{c/e} \times v^i_{c/e}||}, \qquad \hat{T}^i_r = \hat{N}^i_r \times \hat{R}^i_r
\end{equation}
Here, $r^i_{c/e}$ and $v^i_{c/e}$ are the position and velocity vectors of the chief/leader relative to the Earth's center expressed in the inertial frame. The transformation between ECI and RTN frames is given by:
\begin{equation}
    R^i_r = \begin{bmatrix}
        \hat{R}^i_r & \hat{T}^i_r & \hat{N}^i_r
    \end{bmatrix}, \qquad R^r_i = (R^i_r)^T = \begin{bmatrix}
        (\hat{R}^i_r)^T \\
        (\hat{T}^i_r)^T \\
        (\hat{N}^i_r)^T
    \end{bmatrix}
\end{equation}  


\section{Orbital Perturbations}

\subsection{Spherical Harmonics} \label{sec:J2}

Gravitational potential with J2 - J4 from \cite{vallado2001, Montenbruck2000}

\begin{equation}
    U(r, \phi) = \frac{\mu}{r} \left[ 1 - \sum_{n=2}^{4} J_n \left( \frac{R_E}{r} \right)^n P_n(\sin \phi) \right]
\end{equation}

where:
\begin{itemize}
    \item $\mu$ is Earth's gravitational parameter
    \item $R_E$ is the Earth's equatorial radius
    \item $r = ||r||$ is the geocentric distance
    \item $\phi$ is the geocentric latitude
    \item $J_n$ are the zonal harmonic coefficients
    \item $P_n(\cdot)$ are the Legendre polynomials
\end{itemize}

\begin{equation}
    a_{grav} = -\nabla U(r)
\end{equation}

TODO: Explain the decreasing significance of J2, J3 and J4 terms, and how they affect the mean orbital elements
TODO: Add a table giving the numerical values for the J2-J4 terms

\subsection{Atmospheric Drag} \label{sec:Atmospheric_Drag}

Atmospheric drag exponential density model from \cite{vallado2001, Montenbruck2000, Roscoe2014}
\[
\mathbf{a}_{\text{drag}} = -\tfrac{1}{2} C_D \, \frac{A}{m} \, \rho \, v_{\text{rel}}^{2} \, \hat{\mathbf{v}}_{\text{rel}},
\]

\begin{itemize}
    \item $C_D$ is the drag coefficient,
    \item $A$ is the effective cross-sectional area,
    \item $m$ is the spacecraft mass,
    \item $\rho$ is the atmospheric density at the spacecraft location,
    \item $\mathbf{v}_{\text{rel}}$ is the velocity of the spacecraft \textit{relative to the atmosphere},
    \item $v_{\text{rel}} = \lVert \mathbf{v}_{\text{rel}} \rVert$,
    \item $\hat{\mathbf{v}}_{\text{rel}} = \mathbf{v}_{\text{rel}} / v_{\text{rel}}$.
\end{itemize}

% Exponential density model
\[
\rho(h) = \rho_0 \exp\!\left( -\frac{h - h_0}{H} \right),
\]

where

\begin{itemize}
    \item $h$ is the altitude, $h = r - R_E$,
    \item $\rho_0$ is the reference density at altitude $h_0$,
    \item $H$ is the scale height.
\end{itemize}

% Full drag acceleration with exponential atmosphere
\[
\mathbf{a}_{\text{drag}}
= -\tfrac{1}{2} C_D \, \frac{A}{m} \, \rho_0
\exp\!\left( -\frac{h - h_0}{H} \right)
v_{\text{rel}}^{2} \, \hat{\mathbf{v}}_{\text{rel}}.
\]




\subsection{Solar Radiation Pressure} \label{sec:Solar_Pressure}

Radiation is modeled by using solar flux at one astronomical unit and scaling by distance from the sun relative to 1 AU. The solar flux at one AU is taken as:

\begin{equation}
    SF_{AU} = 1372.5398 \; \left[ \frac{W}{m^2} \right]
\end{equation}

The cannonball model assumes the spacecraft is a simple sphere. The radiation pressure at 1AU, pSR, can be taken as the solar flux divided by the speed of light.

\begin{equation}
    p_{SR} = \frac{SF_{AU}}{c} \; \left[ \frac{N}{m^2} \right]
\end{equation}

Then, a “scaling factor” can be determined. This “scaling factor” is equivalent to the magnitude of the solar radiation force divided by the distance between the spacecraft and the sun:

\begin{equation}
    \frac{|\mathbf{F}_{\text{radiation}}|}{|\mathbf{r}_{\text{sun}}|} = 
    - c_R \, p_{SR} \, A_d \, \frac{AU^2}{|\mathbf{r}_{\text{sun}}|^3}
    \; \left[ \frac{N}{m} \right]
\end{equation}

rsun is the vector from the spacecraft to the sun in the spacecraft body frame and cR is the reflectivity. This factor is then multiplied by the position vector from the spacecraft to the sun to get the force on the spacecraft due to solar radiation pressure.

\begin{equation}
    \mathbf{F}_{\text{radiation}} =
    \frac{|\mathbf{F}_{\text{radiation}}|}{|\mathbf{r}_{\text{sun}}|}
    \, \mathbf{r}_{\text{sun}} \; [N]
\end{equation}


\subsection{Third-body Perturbations}

Third-body gravitational pull from the sun and the moon. Formulations gathered from \cite{vallado2001, Montenbruck2000, Wertz1978, Roscoe2014}

Let $\mathbf{r}$ be the position of the spacecraft with respect to the Earth, and 
$\mathbf{r}_{3B}$ the position of the third body (Sun or Moon) with respect to the Earth, 
both expressed in the same inertial frame.

The standard third-body acceleration is

\[
\mathbf{a}_{3B}
    = \mu_{3B}
    \left(
        \frac{\mathbf{r}_{3B} - \mathbf{r}}{\lVert \mathbf{r}_{3B} - \mathbf{r} \rVert^{3}}
        \;-\;
        \frac{\mathbf{r}_{3B}}{\lVert \mathbf{r}_{3B} \rVert^{3}}
    \right),
\]

where

\begin{itemize}
    \item $\mu_{3B}$ is the gravitational parameter of the third body (Sun or Moon),
    \item the first term is the gravitational acceleration of the third body on the spacecraft,
    \item the second term subtracts the acceleration of the Earth due to the third body 
          (so the result is in the Earth-centred frame).
\end{itemize}

If you model both Sun and Moon, the total third-body acceleration is

\[
\mathbf{a}_{3B,\text{total}} = \mathbf{a}_{\text{Sun}} + \mathbf{a}_{\text{Moon}},
\]

each term computed using the same formula above with the appropriate 
$\mu_{\odot}$, $\mu_{\text{Moon}}$ and ephemerides.



\section{Numerical Orbit Propagation}
% How basilisk propagates the orbit


\section{Analytical Orbit Propagation - SGP4}
% How 