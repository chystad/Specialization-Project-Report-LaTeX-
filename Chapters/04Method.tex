This chapter describes the method used to investigate the differences between the semi-analytical Skyfield SGP4 orbital propagator and a high-fidelity numerical propagator for satellites in \gls{leo}. The comparison is performed through a series of simulation-based experiments designed to assess long-term propagation accuracy, sensitivity to perturbation modelling and numerical integration effects. The main goal of this chapter is therefore to present a transparent overview of the simulation environment and the comparison methodology. 

A simulation-based approach is adopted for two main reasons. First, the project aims to understand the Skyfield and Basilisk divergence by quantifying individual perturbation effect's impact on predictions and assessing numerical effects. This research question is inherently rooted in simulations, and the comparison in a simulation environment enables systematic variation of conditions to investigate this. Second, in the event that the SGP4 propagation tool is deemed too inaccurate for downstream orbital or formation flight analyses, an alternative propagation tool has to be implemented in its stead. It therefore makes sense to design [WRITE MORE AFTER COMPLETING THE REST OF THE METHODOLOGY CHAPTER]

1. it is useful to compare the SGP4 against a tunable simulation tool in order to get a better understanding of why SGP4 predictions drift over time. This would be harder to deduce from experimental data alone
2. In the event that the SGP4 propagation tool is deemed too inaccurate for downstream orbital or formation flight analyses, an alternative simulator has to be implemented in its stead. It therefore makes sense to design an 




\clearpage
\section{Literature Search}
\label{sec:Literature_Search}
The literature review was conducted to build a solid theoretical foundation for the project, and to map out other similar studies. More specifically, the primary objectives were to identify established propagation methods, understand the major orbital perturbations with their relative impact on spacecrafts in \gls{leo}, and to review existing studies on prediction error growth of SGP4 and numerical orbital propagators over longer time horizons. Based on these objectives, the literature search and following review was structured around three main themes: 
\begin{itemize}
    \item Orbital propagation methods
    \item Orbital perturbation modelling
    \item Prediction error growth 
\end{itemize}
The following subsections will present the structured approach used to collect sources and filter out irrelevant or questionable publications for the literature review. This process will hereby be referred to as \textit{literature search}, and must not be confused with the literature review in chapter \ref{chap:literature_review} that analyses and discusses the sources provided by the search. 

\subsection{Sources \& Search Strategy}
The literature search was initially focused on peer-reviewed journal articles accessed through Scopus and relevant papers provided through academic supervision. As the project matured and new relevant subjects became apparent, the search was later expanded to include books, master's theses and additional journal publications through Google Scholar. For astrodynamical models and perturbation effects, two highly credited textbooks were used extensively, particularly \textit{Satellite orbits: Models, Methods and applications} by Montenbruck and Gill, and \textit{Fundamentals of astrodynamics and applications} by Vallado. These were treated as primary sources and provided the theoretical foundation for the literature review. 
% I deliberately did not mention Skyfield and Basilisk documentation here, as it was not used in the literature review. However, they gave valuable insight into the inner workings of the simulators. I could have written something like "Documentation for Skyfield and Basilisk software frameworks was also consulted. These were necesary to verify how the propagation methods and force modeles are implemented in practice, not just through abstract function calls"


The literature search was conducted by defining a structured set of keywords and search phrases within each of the main themes from Section \ref{sec:Literature_Search}. As familiarity with the field grew, keywords and search phrases were refined and combined into boolean search strings to focus the search towards specific topics. A comprehensive table of all the keywords, search phrases and time ranges used in the literature search is shown in Table \ref{tab:search}. In this context, \textit{time range} refers to an interval that restricts the search to sources published within the interval. This method is applied in order to ensure that the literature is both relevant and up to date. For both \textit{Orbital propagation methods} and \textit{Orbital perturbation modelling} fields, the time range is set to all sources published after 1960. The large time span was chosen by necessity to include older foundational work because many core astrodynamical models and formulations come from early spaceflight-era literature, as explained in Chapter \ref{chap:literature_review}. The time range also served to exclude some of the earliest astrodynamic publications when the field had not yet reached maturity. The \textit{Prediction error growth} field has a narrower time range, specifically after 2010. This has been chosen to exclude older publications that use outdated numerical models and to focus on the newest accuracy evaluation of the SGP4 propagator.


% \begin{table}[H]
% \begin{tabular}{lcll}
% \toprule
% \textbf{Field} & \textbf{Keywords} & \textbf{Search Phrases} & \textbf{Time range} \\ \midrule
% Orbital propagation methods    &                   & \begin{tabular}[c]{@{}l@{}}Satellite orbit\\ Orbit propagation\\ Orbit propagation method\\ Orbital dynamics \\ Astrodynamics\\ Orbital mechanics\\ Numerical orbit integration\\ Analytical orbit propagation\\ Semi-analytical orbit propagation\\ SGP4\\ Cowell's method \\ Runge-Kutta orbit propagator\\ High-fidelity orbit simulation\\ Propagator validation\end{tabular} & 1960 - now          \\ \hline
% Orbital perturbation modelling &                   & \begin{tabular}[c]{@{}l@{}}Orbital perturbation\\ LEO perturbation\\ Gravity field modelling\\ Spherical harmonics\\ Atmospheric drag model\\ LEO drag\\ Exponential atmosphere\\ Solar radiation pressure\\ Cannonball model\\ 3rd body\\ Formation flying perturbation\\ Perturbation comparison\end{tabular}                                                                   & 1960 - now          \\ \hline
% Prediction error growth        &                   & \begin{tabular}[c]{@{}l@{}}Orbit prediction horizon\\ SGP4 accuracy\\ Propagation uncertainty\\ Long-term orbit accuracy\\ Long-term prediction error\\ Numerical integration error\\ Variable-step integrator error\\ Fixed-step integration error\\ Relative orbital motion accuracy\end{tabular}                                                                               & 2010 - now          \\ \hline
% \end{tabular}
% \end{table}


\begin{table}[H]
\begin{tabular}{llll}
\toprule
\textbf{Field} & \textbf{Keywords} & \textbf{Search Phrases} & \textbf{Time range} \\ \midrule
Orbital propagation methods  & \begin{tabular}[t]{@{}l@{}}SGP4\\ Propagation\\ Astrodynamics\\ Orbit\\ Runge-Kutta\\ Satellite\\ LEO\end{tabular}                   & \begin{tabular}[t]{@{}l@{}}Satellite orbit\\ Orbit propagation\\ Orbit propagation method\\ Orbital dynamics \\ Astrodynamics\\ Orbital mechanics\\ Numerical orbit integration\\ Analytical orbit propagation\\ Semi-analytical orbit propagation\\ Cowell's method \\ Runge-Kutta orbit propagator\\ High-fidelity orbit simulation\\ Propagator validation\end{tabular} & 1960 - now          \\ \hline
Orbital perturbation modelling & \begin{tabular}[t]{@{}l@{}}Perturbation\\ Disturbance\\ Gravity\\ Model\\ LEO\\ SRP\\ Drag\end{tabular}                              & \begin{tabular}[t]{@{}l@{}}Orbital perturbation\\ LEO perturbation\\ Gravity field modelling\\ Spherical harmonics\\ Atmospheric drag model\\ LEO drag\\ Exponential atmosphere\\ Solar radiation pressure\\ Cannonball model\\ 3rd body\\ Formation flying perturbation\\ Perturbation comparison\end{tabular}                                                            & 1960 - now          \\ \hline
Prediction error growth        & \begin{tabular}[t]{@{}l@{}}Simulation\\ Accuracy\\ Error\\ Prediction\\ SGP4\\ Fixed-step\\ Variable-step\\ Integration\end{tabular} & \begin{tabular}[t]{@{}l@{}}Orbit prediction horizon\\ SGP4 accuracy\\ Propagation uncertainty\\ Long-term orbit accuracy\\ Long-term prediction error\\ Numerical integration error\\ Variable-step integrator error\\ Fixed-step integration error\\ Relative orbital motion accuracy\end{tabular}                                                                        & 2010 - now          \\ \hline
\end{tabular}
\caption{Keywords, search phrases and limited time range used in the literature search for each main theme.}
\label{tab:search}
\end{table}


\subsection{Initial Relevance Screening}
Immediate search results were subject to an initial screening to assess their relevance to the project. In this screening process, titles, abstracts and conclusions were reviewed in order to exclude any sources not directly relevant for LEO orbit propagation, perturbation modelling or long time horizon prediction error. The initial search returned approximately 150 sources, which was reduced to 22 remaining sources after the initial relevance screening. The large reduction reflects the broad keyword searches, and the relatively narrow focus of this project. 


\subsection{Quality and Credibility Screening}
Following the initial relevance screening, a second screening phase was conducted to assess the quality and credibility of the remaining sources. Each source's credibility was assessed through a total of four criteria. First, the author's expertise and standing in the field is evaluated by checking their previous work. This also helps to distinguish foundational contributors from aiding authors. Second, the reputation of the publishing journal or institution is considered. However, even renowned authors and publishers can still publish flawed work. That is why the two next criteria aim to evaluate their content in more detail. The third criteria detect errors and inconsistencies by cross-checking facts and mathematical formulations. Here, the primary sources from Montenbruck and Vallado \cite{Montenbruck2000, vallado2001} was especially important to evaluate astrodynamical models, but other previously assessed independent sources were also used for the same purpose. Finally, the source's methodology and overall structure were evaluated to ensure that final results were produced by using a well thought out approach. After the final screening, the number of remaining sources had been reduced to 13

\begin{comment}
Should contain the following:
    * How you preformed the search
        - Which databases you used (Scopus, Google Scholar, IEEE, etc.)
        - How you constructed search strings (AND, OR, NOT)
        - What time ranges you set
        - How you filtered results
        - How many articles you ended up with
        - How you grouped them
\end{comment} 





\section{Simulation}
In order to compare the Skyfield SGP4 predictions against a high-fidelity Basilisk counterpart, a modular simulation framework has been developed in Python. The following subsections will present the high-level structure of this simulator, a deeper understanding of each orbit propagation sub-simulation and its implementation, the data processing methodology, and finally, how the simulator is configured to run specific simulation cases. For future reference, the term \textit{simulator} describes the simulation framework in its entirety, and \textit{sub-simulation} will be used to denote an individual orbital propagator within it. 

\subsection{Overall Simulation \& Analysis Pipeline}
\begin{figure}[H]
    \centering
    \includegraphics[width=0.7\textwidth]{Figures/04Method/Modular simulation.jpg}
    \caption{High-level simulator flow diagram} 
    \label{fig:high_level_program_flow} 
\end{figure}
The overall structure of the simulation framework is illustrated in Figure \ref{fig:high_level_program_flow}, which presents a high-level abstraction of the simulator as a sequence of modules. The simulator is composed of four modules: a configuration module, a semi-analytical orbit propagation module based on Skyfield SGP4, a numerical orbit propagation module based on Basilisk, and a data processing and plotting module. A more detailed view of the internal processes executed within each module, from configuration up to and including the Basilisk simulation module, is shown in Figure \ref{fig:program_flow}. This diagram is divided into five separate layers meant to represent different aspects of the simulator structure. The top and bottom layers represent the simulator inputs and outputs, respectively. The middle layer show the modules divided into program blocks, and its adjacent layers show how object instances are passed between them. Finally, the \textit{Data processing and plotting} module is omitted from this diagram to avoid unnecessary complexity, and because it is explained in greater detail in Section \ref{sec:data_processing}.

\begin{figure}[H]
    \centering
    \includegraphics[width=1.0\textwidth]{Figures/04Method/Simulator Flow final.png}
    \caption{Detailed simulator flow diagram} 
    \label{fig:program_flow} 
\end{figure}

The configuration module is responsible for initializing the global simulation environment. Three user-defined configuration files specify the simulation start time and duration, satellite properties and \gls{tle}s, numerical integrator settings and perturbation models. These configuration files are parsed and combined into a single configuration instance that is shared by both Skyfield and Basilisk sub-simulations to ensure consistency between the propagators. As part of this process, a snapshot of the combined configuration is outputted as \texttt{<time>\_cfg.yaml}, where \texttt{<time>} denotes real-world timestamp for when the simulator was executed. This snapshot is important for traceability because it enables checking what simulation configuration was used to produce a set of simulator outputs at a later stage. 

Following configuration, the Skyfield SGP4 sub-simulation is initialized and executed. Using the specified \gls{tle}s and simulation parameters, Skyfield propagates the satellite states over the full simulation duration. In this process, satellite translational states are evaluated at discrete time intervals defined by the simulation configuration, and the internal workings of the SGP4 algorithm. After completion, the resulting \gls{eci} position and velocities are outputted as a data file named \texttt{<time>\_skf.h5}, where \texttt{<time>} is the same timestamp used by the configuration module. The Skyfield SGP4-propagated satellite states at the initial simulation time are extracted and used to define the initial conditions for the Basilisk sub-simulation. This step is necessary because Basilisk does not natively support \gls{tle}-based initialization, and this approach ensures that both propagation methods start from identical initial states. 

Next, the Basilisk sub-simulation is initialized. Compared to Skyfield, this initialization involves additional steps, including the creation of simulation processes, tasks, spacecraft and recorders required by the Basilisk framework, as well as defining numerical integrators and environmental models. The inclusion of individual perturbation models such as Earth's spherical harmonics, atmospheric drag, solar radiation pressure, and third-body gravitational effects, is controlled by the global configuration. In the detailed flow diagram in Figure \ref{fig:program_flow}, these conditional steps are indicated by the annotation ''if Config'', showing that these models will only be created if specified by the simulator configuration.  Once initialized, the Basilisk simulation is run, and the resulting satellite \gls{eci} state trajectories are outputted as a file named \texttt{<time>\_bsk.h5}.

The final module in the pipeline is the data processing and plotting module. In Figure \ref{fig:high_level_program_flow}, the flow from the Basilisk simulation to this module is denoted by a dashed arrow. This notation indicates that the module can either be executed directly as part of the simulation run, or be called independently at a later time. When executed in series with the other modules, it loads the newly generated Skyfield and Basilisk output files and performs the data processing required to generate the plots specified in the global configuration. Depending on the requested output, this may involve simple visualization of position and velocity data, or additional data processing steps such as reference frame transformations for relative motion analysis. The data processing specifics are presented in greater detail in Section \ref{sec:data_processing}. Alternatively, the same module can be configured to operate on previously generated simulation data, allowing analyses to be repeated or extended without running the computationally expensive propagations. 



\begin{comment}
Should explain how the simulation works
* Inputs
* Outputs
* Processing steps
* In what order these occour
\end{comment}



\subsection{Orbit Propagation Methods}


\subsection{Simulation Configuration \& Assumptions}


\subsection{Data Processing \& Comparison Metrics}
\label{sec:data_processing}


\subsection{Verification \& Consistency Checks}


\section{Methodological Limitations}




\begin{comment}


\section{Simulation Design}





 

\section{Software Development} \label{sec:Software_Development}

\subsection{Software Architecture} \label{sec:Software_Architecture}
This section is just for me to organize my thoughts while I work on developing the software architrecture. 

\subsubsection{Main Pipeline}
Sequential steps executed by the main program. 
\begin{enumerate}
    \item Initialize the simulation environment with global satellite instances and simulation parameters from config.
    \item Call a function to run the SGP4 propagator from ''GNSS\_R\_System\_Simulator''.
    \begin{enumerate}
        \item Its initial conditions are the TLEs and the initial epoch, fetched from the satellite instances (from config).
        \item This function call also extracts the states ($\vec{r}$ and $\vec{v}$) at the initial epoch in the SGP4 output frame (TEME, I believe).
        \item The function call then transforms the states to the Basilisk inertial frame (TEME(?) $\rightarrow$ ECI(?)) and adds it to the satellite instances.
        \item Log the output trajectories in an external file or standardized data structure for later use.
    \end{enumerate}
    \item Call a function to run the dynamical Basilisk simulation. \begin{enumerate} 
        \item Its initial conditions are the states from the SGP4 output at the initial epoch, fetched from the satellite instances.
        \item Log the output trajectories in an external file or standardized data structure for later use.
    \end{enumerate}
\end{enumerate}

\subsubsection{Config}
This subsection describes all the fields included in the config file
\begin{itemize}
    \item SIMULATION (This will contain all the high-level simulation parameters that are global for all simulation frameworks) \begin{itemize}
        \item \texttt{startTime} (Real-world date and time of the simulation start. Is this different from ''inital epoch''?)
        \item \texttt{simulationDuration} (How long period into the future the simulation will simulate)
    \end{itemize} 
    \item SATELLITE (This will contain all the satellite-specific parameters, and there will be one such section for each satellite in the simulation) \begin{itemize}
        \item \texttt{satelliteName} (A string identifier for the satellite)
        \item \texttt{TLELine1} (The first line of the TLE data for the satellite)
        \item \texttt{TLELine2} (The second line of the TLE data for the satellite)
        \item \texttt{initialEpoch} (The epoch time corresponding to the TLE data, in a standardized format. Should always be the same as \texttt{startTime}???)
        \item physical attributes like size, mass, drag coefficient, inertia, etc. (TBD!)
        \end{itemize}
    \item PLOTTING (This will contain all the parameters related to how the simulation results should be visualized) \begin{itemize}
        \item \texttt{savePath} (Path to save the generated plots)
        \item Other plotting parameters (TBD!)
    \end{itemize}
    \item SKYFIELD (This will contain all the parameters specific to the Skyfield simulation) \begin{itemize}
        \item \texttt{timeStep} (The time step for the Skyfield simulation)
        \item Skyfield-specific parameters (TBD!)
    \end{itemize}
    \item BASILISK (This will contain all the parameters specific to the Basilisk simulation framework) \begin{itemize}
        \item \texttt{timeStep} (The time step for the Basilisk simulation)
        \item \texttt{solver} (The numerical solver to be used in the Basilisk simulation)
        \item \texttt{gravityModel} (The gravity model to be used in the Basilisk simulation)
        \item \texttt{radiationPressureModel} (The radiation pressure model to be used in the Basilisk simulation)
        \item Other Basilisk-specific parameters (TBD!)
    \end{itemize}
\end{itemize}



The below list shows additional capabilities in Basilisk that have yet to be implemented, but could serve to increase the fidelity of the simulation.
\begin{itemize}
    \item Track a point relative to the satellite's COM
\end{itemize}


\end{comment}