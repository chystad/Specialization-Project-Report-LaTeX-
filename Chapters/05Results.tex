% \section{Results overview and roadmap}

This chapter presents the results generated by performing the three numerical experiments described in the Methodology chapter and summarized in Table \ref{tab:experiments_overview}. First, a benchmark comparison between the most accurate Basilisk configuration and the Skyfield \gls{sgp4} solution is presented based on data gathered in experiment \#1. In this context, the most accurate Basilisk configuration refers to the simulation including all modeled perturbation effects and using the highest-accuracy numerical integrator considered in this study. These results establish both the magnitude of the divergence between the two propagators and how this divergence manifests in formation relative satellite positions.

Second, the sensitivity of the propagation results to individual perturbation models is investigated by presenting data from experiment \#2. Earth-relative and formation-relative comparisons are presented in pairs, first relative to the Skyfield \gls{sgp4} baseline and subsequently relative to a reduced Basilisk perturbation configuration. This allows the relative impact of dominant and secondary perturbations to be assessed.

Finally, using data from experiment \#3, the influence of numerical integration settings within Basilisk is examined. These results quantify the contribution of numerical propagation uncertainty to the overall divergence and provide additional context for interpreting the differences observed between the two propagators.

\clearpage
\section{Benchmark Comparison}

This section presents a benchmark comparison between the Skyfield \gls{sgp4} propagator and the Basilisk configuration including all modeled perturbations and the highest-accuracy numerical integrator. All figures shown in this section are produced exclusively from data generated in Experiment \#1.

\subsection{Earth-Relative Position Difference}

Figure \ref{fig:ex1_groundtrack} and Figure \ref{fig:ex1_eci_diff} together illustrate the Earth-relative position divergence between the Basilisk and Skyfield propagation outputs. Figure \ref{fig:ex1_groundtrack} shows the leader satellite's trajectory projected on to the Earth's surface after approximately seven days propagation time. By this time, the closest distance between the resulting ground tracks are approximately $50 \text{km}$.
\begin{figure}[H]
    \centering
    \includegraphics[ width=0.75\textwidth]{Figures/05Results/Ex1/Ex1 - Groundtrack comparison - RKF78.png}
    \caption{Projected ground track comparison for the leader satellite obtained using the Skyfield \gls{sgp4} propagator and the most accurate Basilisk configuration, shown over a selected time window near the end of the propagation interval at seven days.}
    \label{fig:ex1_groundtrack}
\end{figure}
Figure \ref{fig:ex1_eci_diff} shows the Earth-relative position difference magnitude between the Basilisk and Skyfield propagators as a function of time, computed relative to the Skyfield \gls{sgp4} baseline. The rate of divergence between the two solutions increases with time, reaching approximately $750 \text{km}$ after one week of propagation.
\begin{figure}[H]
    \centering
    \includegraphics[width=0.75\textwidth]{Figures/05Results/Ex1/Ex1 - Leader simulator position difference magnitude.png}
    \caption{Leader satellite \gls{eci} position difference magnitude between the Skyfield \gls{sgp4} baseline and the most accurate Basilisk configuration as a function of time.}
    \label{fig:ex1_eci_diff}
\end{figure}

Despite the large Earth-relative position difference magnitude shown in Figure \ref{fig:ex1_eci_diff}, the shortest distance between the projected ground tracks in Figure \ref{fig:ex1_groundtrack} remains comparatively small. This indicates that the divergence between the two propagation solutions accumulates primarily in the along-track direction over the course of the propagation interval, leading to a large inertial separation without a correspondingly large cross-track displacement on the Earth's surface.

\subsection{Formation-Relative Position Difference}

Figure \ref{fig:ex1_rtn_diff} presents the sub-simulation difference in the relative position between the leader and follower satellites expressed in the \gls{rtn} frame and computed relative to the Skyfield \gls{sgp4} baseline.

\begin{figure}[H]
    \centering
    \includegraphics[ width=0.75\textwidth]{Figures/05Results/Ex1/Ex1 - Leader-follower relative position simulator difference - no radial.png}
    \caption{Sub-simulation difference in the relative position between the leader and follower satellites in the \gls{rtn} frame. The most accurate Basilisk configuration is compared against the Skyfield \gls{sgp4} baseline.}
    \label{fig:ex1_rtn_diff}
\end{figure}

The along-track component exhibits a near-linear increase in the sub-simulation difference over time. The omitted radial component is tightly coupled to the along-track component showing the same trend, while the cross-track component remains bounded and is dominated by relatively small periodic variations. These results demonstrate that the divergence between the two propagators manifests differently depending on the relative-motion component considered, even when both satellites are initialized with identical relative states.




\section{Sensitivity to Perturbation Configurations}
This section investigates how individual perturbation models affect the propagated satellite positions. Earth-relative and formation-relative positions are presented together to highlight similarities and differences between absolute and relative motion behavior. All results presented in this section are derived from data generated in Experiment \#2.

\subsection{Perturbation Impact on Earth-Relative Position}
Just like Figure \ref{fig:ex1_eci_diff}, Figure \ref{fig:ex2_eci_big} shows the Earth-relative position difference magnitude for the leader satellite, but includes additional Basilisk perturbation configurations. In this figure, only the dominant perturbation configurations are included, namely spherical harmonic gravity models of order two, three, and four, as well as the configuration with all perturbations enabled. Less influential perturbations are omitted from this figure to reduce visual clutter and because their individual effects are difficult to distinguish on this scale. Their relative impact is instead examined in Figure \ref{fig:ex2_eci_small}. As shown, variations in the order of the spherical harmonic gravity model produce the largest changes in the Earth-relative position difference magnitude. This is evident by the large difference between the 2nd and 4th order spherical harmonic configurations, compared to the barely noticeable difference between the 4th order and the all-inclusive configurations. Increasing the gravity model order leads to progressively larger divergence relative to the Skyfield \gls{sgp4} baseline over the propagation interval.
\begin{figure}[H]
    \centering
    \includegraphics[width=0.75\textwidth]{Figures/05Results/Ex2/Ex2 - Leader simulator position difference magnitude - Big perturb comparison.png}
    \caption{Leader satellite \gls{eci} position difference magnitude between the Skyfield \gls{sgp4} baseline and selected Basilisk perturbation configurations, including spherical harmonic gravity models of order two, three, and four, as well as the configuration with all perturbations enabled.}
    \label{fig:ex2_eci_big}
\end{figure}
To isolate the relative contributions of smaller perturbations, Figure \ref{fig:ex2_eci_small} presents the \gls{eci} position difference magnitude computed relative to a fourth-order spherical harmonic Basilisk baseline. In this representation, the individual effects of atmospheric drag, solar radiation pressure, and third-body gravitational perturbations from the Sun and Moon become distinguishable.

\begin{figure}[H]
    \centering
    \includegraphics[ width=0.75\textwidth]{Figures/05Results/Ex2/Ex2 - Leader simulator position difference magnitude - Small perturb.png}
    \caption{Leader satellite \gls{eci} position difference magnitude relative to a fourth-order spherical harmonic Basilisk baseline, showing the isolated effects of atmospheric drag, solar radiation pressure, and third-body perturbations from the Sun and Moon.}
    \label{fig:ex2_eci_small}
\end{figure}

The perturbations introduce both secular and periodic components in the Earth-relative position difference, with magnetudes significantly smaller than those observed in the Skyfield-relative comparison. This figure isolates the relative contributions of secondary perturbations that are otherwise obscured when compared directly against the Skyfield \gls{sgp4} baseline. Of the smaller perturbations, Figure \ref{fig:ex2_eci_small} clearly shows that third-body perturbation from the Moon has the highest impact, followed by the Sun, \gls{srp} and drag in descending order of significance. 

\subsection{Perturbation Impact on Formation-Relative Position}

Figure \ref{fig:ex2_rtn_sh} presents the \gls{rtn} position sub-simulation difference between the leader and follower satellites computed relative to the Skyfield \gls{sgp4} baseline, just like Figure \ref{fig:ex1_rtn_diff}. The plotted Basilisk configurations correspond to the same spherical harmonic gravity model variations shown in Figure \ref{fig:ex2_eci_big}. The radial component is omitted for clarity.

\begin{figure}[H]
    \centering
    \includegraphics[ width=0.75\textwidth]{Figures/05Results/Ex2/Ex2 - Leader-follower relative position simulator difference - SH comparison - no radial.png}
    \caption{Sub-simulation differences in the relative position between the leader and follower satellites in \gls{rtn} frame. Multiple Basilisk configurations using spherical harmonic gravity models of different orders are compared against the Skyfield \gls{sgp4} baseline.}
    \label{fig:ex2_rtn_sh}
\end{figure}

As in the Earth-relative case, variations in the spherical harmonic gravity model dominate the along-track sub-simulation difference. An interesting observation is that the second-order spherical harmonic Basilisk configuration remains comparatively close to the Skyfield \gls{sgp4} baseline. The cross-track component remains bounded and exhibits periodic behavior. Secondary perturbations are visually indistinguishable from the full-perturbation configuration on this scale and are therefore omitted. Figure \ref{fig:ex2_rtn_all} presents the smaller perturbations' impact on the propagation by comparing the \gls{rtn} leader-follower relative position against a forth-order spherical harmonic Basilisk configuration baseline. This figure uses the same RTN representation as Figures \ref{fig:ex1_rtn_diff} and \ref{fig:ex2_rtn_sh}, differing only in the choice of baseline and included perturbation configurations.

\begin{figure}[H]
    \centering
    \includegraphics[ width=0.75\textwidth]{Figures/05Results/Ex2/Ex2 - Leader-follower relative position simulator difference - All perturb comparison - no radial.png}
    \caption{Formation-relative position sub-simulation difference in the RTN frame relative to a fourth-order spherical harmonic Basilisk baseline, showing the isolated effects of atmospheric drag, solar radiation pressure, and third-body perturbations from the Sun and Moon. The radial component is omitted for clarity.}
    \label{fig:ex2_rtn_all}
\end{figure}

In this representation, the relative contributions of atmospheric drag, solar radiation pressure, and third-body perturbations from the Sun and Moon become observable in the along-track and cross-track components. The relative significance of the perturbation effects are consistent with those observed in the Earth-relative case, while the absolute magnitudes differ due to the relative-motion formulation.



\section{Sensitivity to Numerical Integration Configuration}

This section investigates how numerical integration settings within Basilisk affect the propagated satellite positions. All results presented in this section are derived from data generated in Experiment~\#3. The purpose of this experiment is to quantify the sensitivity of the propagation results to numerical integration choices and to assess their contribution to the overall divergence observed between the Basilisk and Skyfield \gls{sgp4} propagators.

\subsection{Numerical Impact on Earth-Relative Position}

Figure~\ref{fig:ex3_eci_num} shows the leader satellite \gls{eci} position difference magnitude between the Skyfield \gls{sgp4} baseline and multiple Basilisk configurations using different numerical integration settings. The compared configurations differ in numerical integrator type and step-size, but they are all configured to enable all perturbation models. 

\begin{figure}[H]
    \centering
    \includegraphics[width=0.75\textwidth]{Figures/05Results/Ex3/Ex3 - Leader simulator position difference magnitude - numerical comparison.png}
    \caption{Leader satellite \gls{eci} position difference magnitude between the Skyfield \gls{sgp4} baseline and Basilisk configurations using the numerical integration methods RKF78, RKF45 and RK4, and fixed step sizes ranging between one and fifty seconds.}
    \label{fig:ex3_eci_num}
\end{figure}

Across all numerical integration configurations, the resulting \gls{eci} position difference magnitudes remain closely clustered throughout the propagation interval. While small deviations between integrator settings are observable, these differences are minor compared to the total divergence between Basilisk and the Skyfield \gls{sgp4} solution observed in previous sections. No numerical configuration produces a trajectory that converges toward the Skyfield baseline or becomes numerically unstable over time. Larger fixed step sizes produce larger deviations from the Skyfield baseline, and the RKF78 integrator produces the trajectory closest to it.

\subsection{Numerical Impact on Formation-Relative Position}

Figure~\ref{fig:ex3_rtn_num} presents the sub-simulation difference in the relative position between the leader and follower satellites expressed in the \gls{rtn} frame and computed relative to the Skyfield \gls{sgp4} baseline. The radial component is omitted for clarity.

\begin{figure}[H]
    \centering
    \includegraphics[width=0.75\textwidth]{Figures/05Results/Ex3/Ex3 - Leader-follower relative position simulator difference - numerical comparison - no radial.png}
    \caption{Sub-simulation difference in the relative position between the leader and follower satellites in the \gls{rtn} frame. Multiple Basilisk configurations using different numerical integration settings are compared against the Skyfield \gls{sgp4} baseline.}
    \label{fig:ex3_rtn_num}
\end{figure}

The along-track component again exhibits the largest sub-simulation difference, while the cross-track component remains bounded and dominated by periodic variations. Differences between numerical integration configurations are small relative to the overall divergence between the Basilisk and Skyfield propagators. The qualitative behavior of the relative-motion components remains consistent across all tested numerical integration settings. These results indicate that, within the tested parameter space, numerical integration settings in Basilisk have a limited influence on both absolute \gls{eci} position propagation and formation-relative motion when compared to the effects of force-model selection.


\section{Concluding Remarks}
Across all experiments, the results consistently show that force-model fidelity dominates the divergence between Basilisk and Skyfield SGP4, while numerical integration settings have a comparatively minor influence. The most significant perturbation effect by far the spherical harmonic gravity model, followed by third-body perturbations from the Moon, Sun, \gls{srp} and drag in decreasing order. Increasing physical model fidelity and numerical accuracy within Basilisk does not lead to convergence toward the SGP4 solution. These findings motivate a discussion of the fundamental modeling differences between the two propagation approaches.