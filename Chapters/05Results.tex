\section{Simulation Parameters}
\begin{table}[h!]
\centering
\begin{tabular}{c c c}
\hline
$Parameter$ & $Value$ & $Unit$ \\
\hline
$C_D$ & $1.5-3$ & -\\
$A_D$ & - & - \\
$C_R$ & $-$ & -\\
$A_{SRP}$ & - & - \\
\hline
\end{tabular}
\caption{Physical constants and associated numerical value. $J$-terms courtesy of \cite{Zhong2013}.}
\label{tab:sim_param}
\end{table}

\section{Prelim results}
The current simulator implementation has been used to generate preliminary results for a two-satellite scenario consisting of one chief and one follower spacecraft initialized on different orbital planes in LEO. Skyfield’s SGP4 model serves as the baseline, while the Basilisk simulation includes gravity-gradient effects, solar radiation pressure, and lunar third-body pull. Atmospheric drag has not yet been implemented, and both propagators are run over the same integration time for direct comparison. 

Figure 1 and 2 show the ECI position and velocity differences between Basilisk and SGP4 for the chief and follower spacecraft, respectively. Figure 3 presents the difference in relative motion between the two spacecrafts expressed in the RTN frame. In all cases, the divergence begins small but increases steadily over time. This divergence can be attributed to a combination of intrinsic numerical simulator difference and additional physical perturbations included only in the Basilisk simulation. The large discrepancy visible in the relative-motion plot indicates that a formation-keeping delta-V analysis performed with the two simulation frameworks would yield significantly different results after 24 hours. Further testing and analysis are required to determine how much of this divergence is caused by numerical effects versus physical modelling differences.


Below is a list over all types of plots that I deem valuable for the specialization project. Each plot-type shall be justified by a short description of its usefulness. 

\subsection{Simulator output difference plots}
Show the absolute value logarithmic difference between the simulators. This will make it easier to see the small differences for times smaller than 6 hours, and the big differences after 24 hours. Do this for both the absolute pos/vel difference plot, and the relative formation pos/vel plots. 

\subsection{R} 


\begin{comment}
\section{More figures}

This section includes some examples of different types of figures to include. A simple single figure is shown in figure \ref{fig:trajectory_angle}, while figure \ref{fig:cyclic} shows how three subfigures can be included together. Remember to change both the caption and the title in the square brackets before the caption, which will show up in the list of figures. \\

\noindent page fill page fill page fill page fill page fill page fill page fill page fill page fill page fill page fill page fill page fill page fill page fill page fill page fill page fill page fill page fill page fill page fill page fill page fill page fill page fill page fill page fill page fill page fill page fill page fill page fill page fill page fill page fill page fill page fill page fill page fill page fill page fill page fill page fill page fill page fill.

\begin{figure}[H]
  \centering
  \includegraphics[width=1\textwidth]{Figures/trajectory angle.png}
  \caption[Trajectory angle]{The trajectory angle found for the trajectory ABC between the three points A, B and C. The two different C-points show the angle gotten for relatively unchanged directional trajectory with C', and opposite directional trajectory with C''.}
  \label{fig:trajectory_angle}
\end{figure}



\begin{figure}[H]
  \centering
  \subfloat[Time as a linear variable.]{\includegraphics[width=0.75\textwidth]{Figures/diurnal_values_linear.png}\label{fig:diurnal_lin}}
  \hfill
  \subfloat[Time represented as pairs of sine and cosine.]{\includegraphics[width=0.75\textwidth]{Figures/diurnal_values_sine.png}\label{fig:diurnal_trig}}
  \hfill
  \subfloat[Time as a cyclic feature.]{\includegraphics[width=0.6\textwidth]{Figures/diurnal_values_circle.png}\label{fig:diurnal_cyclic}}
  
  \caption[Time as a cyclic feature]{Figure (a) shows the time variable in the data for the first 100 rows using linear time as minutes past midnight, figure (b) shows the trigonometric time representation for one cycle where each time point has a unique sine and cosine-pair value, and figure (c) shows time as a cyclic feature with the trigonometric pairs.}
\label{fig:cyclic}
\end{figure}
\end{comment}